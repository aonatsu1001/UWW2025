%\documentstyle[a4j,epsbox,graphicx]{jarticle}
\documentclass[a4j]{jarticle}%変更禁止!
%%%%%%%%%%%%usepackageは適宜追加してください.%%%%%%%%%%%%%%%%%%%%%%%%%%%%%%%%%%%%%%%%%%%%%%%%%%%
%\usepackage{epsbox}
%\usepackage{graphicx}
\usepackage[dvipdfmx]{graphicx,color}
\usepackage{comment}
\usepackage{enumitem}
\usepackage{multirow}
%%%%%%%%%%%%%%%%%%%%%%%%%%%ここから変更禁止%%%%%%%%%%%%%%%%%%%%%%%%%%%%%%%%%%%%%%%%%%%%%%%%%%
\topmargin -28mm
\oddsidemargin -15mm
\evensidemargin -15mm
\textwidth 185mm
\textheight 275mm
\columnsep 6mm

%\def\toujitu{Dec. 2020}

\makeatletter
\def\section{\@startsection{section}{2}{\z@}{.8ex plus .8ex minus 
 .2ex}{.05ex plus .07ex}{\large\bf}}
\makeatother
\makeatletter
\def\subsection{\@startsection{subsection}{2}{\z@}{.8ex plus .8ex minus 
 .2ex}{.05ex plus .07ex}{\bf}}
\makeatother


\pagestyle{empty}

\begin{document}

\baselineskip 4.75mm

\twocolumn
[
\footnotesize 
\begin{center}
{~}\\
%\begin{center}
%{ユビキタスウェアラブルワークショップ2021 
%\hfill \toujitu}\\
%%%%%%%%%%%%%%%%%%%%%%%%%%%ここまで変更禁止%%%%%%%%%%%%%%%%%%%%%%%%%%%%%%%%%%%%%%%%%%%%%%%%%%

%%%注意!!\vspaceは図表部分のみ見にくく(醜く)ならない範囲内で使用可能とします.%%%%%%%%%%%%%%%%%%%%%%%%%%%%

\medskip
{\large
%タイトル
{\bf 視点移動情報に基づく視野外への視覚的関心方向の推定}\\
}
\medskip
{\large
%著者 同じ所属の人が連続する場合は連続する同じ所属の著者の最後の著者のみに所属を付けること.
         堀部青夏,  寺田 努, 塚本昌彦(神戸大)
}
\end{center}
]

\section{研究の背景と目的}
近年のHMDの注目度の向上と普及により,HMDの機能である仮想空間上でのディスプレイ表示を活用して,HMDを装着した状態で作業を行う機会が増加する可能性がある.
しかし,HMD装着時の視野の狭さによる作業効率の低下や,首の疲労感が課題として挙げられる.
これらの問題を解決するため,HMD装着時の視野を拡張する研究は数多く存在する.
例として,
村上らの視点の動きに合わせてVR空間上の表示画面をスクロールする手法\cite{bib1}や,
Fanらの背後のオブジェクトの移動などの動的な変化があれば,HMDの表示画面に後方映像を合成して表示する手法\cite{bib2}などがある.
これらの手法は,HMDの装着によって狭窄した視野を回復させるだけでなく,肉眼における視野外領域も見えるようになるなど,
HMD装着前よりも視野を拡張できる.
しかし従来手法では,
画面酔いや,
通常時の視野を妨げる可能性があるなどの課題が存在する.
また首の疲労感もHMD装着時の課題として挙げられるため,首の負担を軽減しつつ,ユーザが注意を向けたいと考える視野外方向の映像を表示する必要がある.
そこで本研究では,視点の動きに合わせて,ユーザが注意を向けたいと考える視野外方向の映像を表示することで,HMD装着時の視野を拡張するシステムを提案する.
本稿では,ユーザが注意を向けたいと考える視野外方向を視覚的関心方向と定義し,
視点移動情報に基づいて視覚的関心方向を推定する手法を提案する.

\section{提案手法}
視点移動情報に基づいて視野外への視覚的関心方向を推定する手法を提案する.
本研究では,試験的に左側の視野外を左上,右上,左下,右下に4分割した領域を,推定する視覚的関心方向の対象とした.
まずは,データセットを作成するために,頭部を固定した状態で,正面の基準点からそれぞれの方向に50回ずつ視点移動を繰り返した際の,
初速や視点方向ベクトルなどの視点移動情報を記録した.
なお,サンプリングレートは60 Hzであった.
データセットは,基準点からそれぞれの方向への視点移動情報のみを用いて作成した.
次に,得られたデータに対してスライディングウィンドウ処理を施しながら,特徴量を抽出した.
1回の視点移動におけるデータ長は約50サンプル (0.83s)であり,5サンプルのウィンドウを1サンプルずつスライドした.
特徴量としては,ウィンドウ内の視点方向ベクトルや瞳孔径の平均値を用いた.
最後に,抽出した特徴量を用いて,視野外への視覚的関心方向を分類する機械学習モデルを構築した.
分類アルゴリズムには,ランダムフォレストを採用した.

\section{実験} 
\subsection{実験内容}
構築した分類モデルによる視野外への視覚的関心方向の推定精度を評価するための実験を行った.
モデルの精度評価には5分割交差検証を用い,各分割における評価結果の平均値を最終的な性能指標とした.
評価指標には正解率や適合率,再現率,F値を用いた.


\subsection{結果と考察}
5分割交差検証の結果を表1に示す.
平均正解率は93.28\%,適合率は93.09\%,再現率は93.11\%,F値は93.08\%であった.
また,各クラスにおける正解率の標準偏差は0.0050と極めて小さい値にとどまっており,
構築した分類モデルが安定して高い精度で視野外への視覚的関心方向を推定できることが示された.
続いて,5分割交差検証における混同行列を図\ref{fig1}に示す.
混同行列の結果から,各クラスに対して高い適合率が得られていることが確認された.
一方で,左側の視野外の右下と左下方向に関しては,適合率がやや低下していることが分かった.
これは,人間は上方向よりも下方向への視点移動の際にノイズが多く発生する可能性を示唆している.
今後は,下方向への視点移動の際に生じるノイズに対応するために特徴量の検討を進める予定である.

\section{まとめと今後の展望}
本稿では,視点移動情報に基づいて視野外への視覚的関心方向の推定手法を提案し,
推定精度を検証するための実験を行った.
今後は,提案手法を取り入れた視野拡張システムを構築し,その有効性を評価する予定である.



 \begin{thebibliography}{10}
\bibitem{bib1}
村上幸乃, 橋本 直: Neckless: アイトラッキングに基づく疑似的な眼球の可動範囲拡張, 情報処
理学会研究報告 (エンターテインメントコンピューティングシンポジウム研究会), Vol. 2022, pp. 84--85 (Aug. 2022).
\bibitem{bib2}
K. Fan, J. Huber, S. Nanayakkara, and M. Inami: SpiderVision: Extending the Human Field of View for Augmented Awareness, \textit{Proc. of the 5th Augmented Human International Conference (AH 2014)} (AH 2014), pp. 1--8 (Mar. 2014).


\end{thebibliography}


% Please add the following required packages to your document preamble:
% \usepackage{multirow}
\begin{table}[t]
\centering
\caption{5分割交差検証の結果}
\vspace{2mm}
\label{tab:accuracy_per_direction}
\resizebox{\columnwidth}{!}{%
    \begin{tabular}{ccccc}
    \hline
    方向 & 正解率 & 適合率 & 再現率 & F値 \\ \hline
    左上 & 0.9902 & 0.9788 & 0.9878 & 0.9833 \\
    右上 & 1.0000 & 1.0000 & 1.0000 & 1.0000 \\
    右下 & 0.9369 & 0.8521 & 0.8662 & 0.8591 \\
    左下 & 0.9385 & 0.8919 & 0.8702 & 0.8809 \\ \hline
    \textbf{平均 $\pm$ 標準偏差} & \textbf{0.9328 $\pm$ 0.0050} & \textbf{0.9309 $\pm$ 0.0053} & \textbf{0.9311 $\pm$ 0.0054} & \textbf{0.9308 $\pm$ 0.0053} \\ \hline
    \end{tabular}%
}
\end{table}

\begin{figure}[!t]
\vspace{2mm}
  \begin{center}
    \includegraphics[width=1\linewidth]{confusion_matrix.pdf}
  \end{center}
    \vspace{-0.8cm}
  \caption{5分割交差検証の混同行列}%{}内にタイトルを記入してください
  \label{fig1}
\end{figure}

\end{document}

%\documentstyle[a4j,epsbox,graphicx]{jarticle}
\documentclass[a4j]{jarticle}%変更禁止!
%%%%%%%%%%%%usepackageは適宜追加してください.%%%%%%%%%%%%%%%%%%%%%%%%%%%%%%%%%%%%%%%%%%%%%%%%%%%
%\usepackage{epsbox}
%\usepackage{graphicx}
\usepackage[dvipdfmx]{graphicx,color}
\usepackage{comment}
\usepackage{enumitem}
\usepackage{multirow}
%%%%%%%%%%%%%%%%%%%%%%%%%%%ここから変更禁止%%%%%%%%%%%%%%%%%%%%%%%%%%%%%%%%%%%%%%%%%%%%%%%%%%
\topmargin -28mm
\oddsidemargin -15mm
\evensidemargin -15mm
\textwidth 185mm
\textheight 275mm
\columnsep 6mm

%\def\toujitu{Dec. 2020}

\makeatletter
\def\section{\@startsection{section}{2}{\z@}{.8ex plus .8ex minus 
 .2ex}{.05ex plus .07ex}{\large\bf}}
\makeatother
\makeatletter
\def\subsection{\@startsection{subsection}{2}{\z@}{.8ex plus .8ex minus 
 .2ex}{.05ex plus .07ex}{\bf}}
\makeatother


\pagestyle{empty}

\begin{document}

\baselineskip 4.75mm

\twocolumn
[
\footnotesize 
\begin{center}
{~}\\
%\begin{center}
%{ユビキタスウェアラブルワークショップ2021 
%\hfill \toujitu}\\
%%%%%%%%%%%%%%%%%%%%%%%%%%%ここまで変更禁止%%%%%%%%%%%%%%%%%%%%%%%%%%%%%%%%%%%%%%%%%%%%%%%%%%

%%%注意!!\vspaceは図表部分のみ見にくく(醜く)ならない範囲内で使用可能とします.%%%%%%%%%%%%%%%%%%%%%%%%%%%%

\medskip
{\large
%タイトル
{\bf 視点移動情報に基づく視野外への視覚的関心方向の推定}\\
}
\medskip
{\large
%著者 同じ所属の人が連続する場合は連続する同じ所属の著者の最後の著者のみに所属を付けること.
         堀部青夏,  寺田 努, 塚本昌彦(神戸大)
}
\end{center}
]

\section{研究の背景と目的}
近年のHMDの注目度の向上と普及により,HMDの機能である仮想空間上でのディスプレイ表示を活用して,HMDを装着した状態で作業を行う機会が増加する可能性がある.
しかし,HMD装着時における視野の狭さによる作業効率の低下が課題として挙げられる.
この問題を解決するため,HMD装着時の視野を拡張する研究は数多く存在する.
例として,早田らの圧縮表示による視野拡張手法\cite{bib1}などが提案されている.
これらの手法は,HMDの装着によって狭窄した視野を回復させるだけでなく,これまで見えなかった領域が見えるようになるなど,
HMD装着前よりも視野情報を提供できる.
しかし従来手法では,視覚的関心方向を必ずしも提示できるものではない.
視覚的関心方向とは,視野外に存在するユーザが注意を向けたいと考える方向のことである.
また,Reed-Jonesらは,人間は視野外に注意を向けたいと考えた際に,視点移動が頭部運動に先行すると示した\cite{bib2}.
これらのことから,視点移動情報から視覚的関心方向を推定できると考えた.

そこで本研究では,視点移動情報から視野外への視覚的関心方向を推定し,視野を拡張するシステムを作成することで,HMD装着時の作業効率の向上を目指す.

\section{提案手法}
本稿では,視点移動情報に基づいて視野外への視覚的関心方向を推定する手法を提案する.
まず左側の視野外を,左上,右上,右下,左下と4分割して,頭部を固定した状態でそれぞれの方向に50回ずつ視点移動を繰り返し,
初速や視点方向ベクトルなどの視点移動情報を記録した.
なお,サンプリングレートは60 Hzであった.
また,ノイズを除去するため,それぞれの方向への移動時にのみラベル付けを行ったデータセットを作成した.
また,一回の移動に対しておよそ50データが得られ,1データをオーバーラップさせた5データずつのウィンドウに分割した.
視点移動の特徴量には,一回の移動におけるウィンドウ内の視点方向ベクトルや瞳孔径を平均して抽出し,
ランダムフォレストを用いて視野外への視覚的関心方向を分類する機械学習モデルを構築した.



\section{実験} 
\subsection{実験内容}
構築した機械学習モデルによる視野外への視覚的関心方向の推定精度を評価するための実験を行った.
データの偏りによる影響を抑えるため,検証方法として5分割交差検証を行い,評価指標には視覚的関心方向の推定がどれほど正しく行えたかを示す正解率を用いた.
具体的には,ラベル付けを行ったデータセットのうちランダムな20\%をテストデータ,残りの80\%を学習データとして扱った.
なお,総サンプル数は7840であった.


\subsection{結果と考察}
オーバーラップウィンドウ処理を適用したデータセットに対し,5分割交差検証を実施したときの正解率を表1に示す.
検証の結果,5回すべての分割において92\%以上の正解率が確認され,
全体としての平均正解率は93.28\%を達成した.
また,各分割における正解率の標準偏差は0.0050と極めて小さい値にとどまっており,
構築した機械学習モデルが安定して高い精度で視野外への視覚的関心方向を推定できることが示された.
続いて,5分割交差検証における混合行列を図\ref{fig1}に示す.
混合行列の結果から,各クラスに対して高い適合率が得られていることが確認された.
一方で,左側の視野外の右下と左下方向に関しては,適合率がやや低下していることが分かった.
これは,人間は上方向よりも下方向への視点移動の際にノイズが多く発生する可能性を示唆している.
今後は,下方向への視点移動の際に生じるノイズに対応するために特徴量の検討を進める予定である.

\section{まとめと今後の展望}
本稿では,視点移動情報に基づいて視野外への視覚的関心方向の推定手法を提案した.
また,構築した機械学習モデルが視野外への視覚的関心方向を高精度で推定できているかを検証した.
今後は,提案手法を取り入れた視野拡張システムを構築し,その有効性を評価する予定である.



 \begin{thebibliography}{10}
\bibitem{bib1}
早田 赳, 岩切宗利: 周辺視野への視角外空間の表示による視野拡張に関する一検討, 情報処理学会研究報告 (グラフィクスとCAD研究会報告), Vol. 2015, No. 10, pp. 1--5 (Feb. 2015).
\bibitem{bib2}
Rebecca J. Reed-Jones, Mark A. Hollands, James G. Reed-Jones, and Lori Ann Vallis: Visually evoked whole-body turning responses during stepping in place in a virtual environment, \textit{Gait \& Posture}, Vol. 30, No. 3, pp. 265--394 (Oct. 2009).


\end{thebibliography}


% Please add the following required packages to your document preamble:
% \usepackage{multirow}
\begin{table}[]
\centering
\caption{5分割交差検証の正解率}
\vspace{2mm}
\label{tab:cv_results_overlapped_jp}
    \begin{tabular}{cc}
    \hline
    Fold & 正解率 \\ \hline
    1 & 0.9279 \\
    2 & 0.9413 \\
    3 & 0.9279 \\
    4 & 0.9318 \\
    5 & 0.9349 \\ \hline
    \textbf{平均 $\pm$ 標準偏差} & \multicolumn{1}{c}{\textbf{0.9328 $\pm$ 0.0050}} \\ \hline
    \end{tabular}%
\end{table}

\begin{figure}[!t]
\vspace{-6mm}
  \begin{center}
    \includegraphics[width=1\linewidth]{confusion_matrix.pdf}
  \end{center}
    \vspace{-1.2cm}
  \caption{5分割交差検証の混合行列}%{}内にタイトルを記入してください
  \label{fig1}
\end{figure}

\end{document}
